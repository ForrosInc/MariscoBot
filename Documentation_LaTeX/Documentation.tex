%Preamble
\documentclass{article}

\usepackage[utf8]{inputenc}
\usepackage[T1]{fontenc}
\usepackage{listings}
\usepackage[margin={2cm,2cm}]{geometry}
\usepackage[document]{ragged2e}

\begin{document}
\title{MariscoBot documentation}
\date{\today{}}
\maketitle{}
\tableofcontents
\newpage
\section{Introduction}
MariscoBot is an open source Discord.js bot. It is under the GNU GPLv3 license. Its function is to communicate with Lord Marisa. It also includes a Youtube audio player, a Danbooru image search command, and a pacman emoticon policy-enforcer.\

It is based upon node.js, discord.js, ytdl-core, and youtube-search. It's currently in an unfinished beta stage.\

\section{Commands}
\begin{flushleft}
\noindent\framebox[1.1\width]{\#help} : Replies with list of MariscoBot commands\newline
\noindent\framebox[1.1\width]{\#Talk} : Replies with loud marisco noises\newline
\noindent\framebox[1.1\width]{\#OUTBEAST} : MariscoBot becomes angry\newline
\noindent\framebox[1.1\width]{\#Marisa} : MariscoBot praises Lord Marisa\newline
\noindent\framebox[1.1\width]{\#Hecatia} : MariscoBot recognizes her natural enemy\newline
\noindent\framebox[1.1\width]{\#OHIO} : MariscoBot greets you\newline
\noindent\framebox[1.1\width]{\#Dango} : To feed Marisa\newline
\noindent\framebox[1.1\width]{\#Spread Marisco} : Spread Marisa's greatness all over the channel\newline
\noindent\framebox[1.1\width]{\#Waifu} : MariscoBot explains who the best waifu is with an image\newline
\noindent\framebox[1.1\width]{\#Mokou} : Posts Mokou memes with questionable translations\newline
\noindent\framebox[1.1\width]{Summon Marisco} : Undocumented magic fuckery\newline
\noindent\framebox[1.1\width]{M-play \$x} : Search \$x on youtube and play the first result\newline
\noindent\framebox[1.1\width]{Danbooru:\$x} : Search for \$x tag on Danbooru \textbf{\emph{WARNING: NSFW}}\newline
\noindent\framebox[1.1\width]{\#8ball \$x} : Marisa answers your questions\par
\end{flushleft}
\section{Configuration}
\subsection{Youtube}
\begin{lstlisting}[frame=single,inputencoding=utf8,showstringspaces=false]
var optsyt = {
  maxResults: 1,
  key: 'YTKeyHere'
};
\end{lstlisting}
\framebox[1.1\width]{maxResults: \$x} should always be 1 except for debugging.\newline
\framebox[1.1\width]{key: \$x} should contain the Youtube Search API key.\par
\subsection{Language}
\begin{lstlisting}[frame=single,inputencoding=utf8,showstringspaces=false]
const SPANISH = 0;
const ENGLISH = 1;
var LANG = SPANISH;
\end{lstlisting}
\framebox[1.1\width]{var LANG = \$x} should be set to a constant representing the array index that your language uses. By default, this is 0 for Spanish and 1 for English.\par
\subsubsection{Language per Guild}
\begin{lstlisting}[frame=single,inputencoding=utf8,showstringspaces=false,linewidth=18.3cm]
//if (msg.guild.id === "guildidhere") let LANG = 1;
if (msg.content === "/dumpguildid") console.log("Guild ID is " + msg.guild.id + "\n");
\end{lstlisting}
By default, this is disabled. If used, it should be configured such that \framebox[1.1\width]{"guildidhere"} is the ID provided by \framebox[1.1\width]{"/dumpguildid"} and \framebox[1.1\width]{LANG} is set to whatever language should be used with that guild instead of the default.\par
\subsection{Discord bot API}
\begin{lstlisting}[frame=single,inputencoding=utf8,showstringspaces=false]
client.login('Discord key Here');
\end{lstlisting}
\framebox[1.1\width]{'Discord key Here'} should be set to your bot's API key.
\end{document}
